\documentclass[a4paper]{article}

\usepackage{INTERSPEECH2021}

% Put the lab number of the corresponding exercise
\title{Language Modelling - Lab 4}
\name{Marco Prosperi (257857)}

\address{
  University of Trento}
\email{marco.prosperi@studenti.unitn.it}

\begin{document}

\maketitle
%
%Dear students, \\
%here you can find a complete description of the sections that you need to write for the mini-report. You have to write a mini-report of \textbf{max 1 page (references, tables and images are excluded from the count)} for each last exercise of labs 4 (LM) and 5 (NLU). \textbf{Reports longer than 1 page will not be checked.} The purpose of this is to give you a way to report cleanly the results and give you space to describe what you have done and/or the originality that you have added to the exercise.
%\\
%\textbf{If you did part A only, you have to just report the results in a table with a small description.}
%
\section{Introduction (approx. 100 words)}
\begin{itemize}
    \item \textit{a summary of what you have done}
\end{itemize}
For the first task of the project the goal was to enhance the performance of the baseline RNN by 
adding incrementally some features.
\section{Implementation details (max approx. 200-300 words)}
Do not explain the backbone deep neural network (e.g. RNN or BERT). Instead, focus on what you did on top of it. \textbf{Add references if you take inspiration from the code of others}

\section{Results}
Add tables and explain how you evaluated your model. Tables and images of plots or confusion matrices do not count in the page limit.
\begin{table}[h!]
  \centering
  \begin{tabular}{lcccc}
      \toprule
      Model & PPL & LR & Hidden & Emb \\
      \midrule
      RNN                     & 173.22 & 0.1    & 100 & 100 \\
      LSTM                    & 137.31 & 2      & 300 & 300 \\
      LSTM + Var Dropout      & 123.14 & 2      & 300 & 300 \\
      LSTM + Var Dropout + AdamW & 109.43 & 0.0001 & 400 & 400 \\
      \bottomrule
  \end{tabular}
  \caption{Perplexity and hyperparameters of the models.}
  \label{tab:results}
\end{table}


\bibliographystyle{IEEEtran}

\bibliography{mybib}
\cite{Rabiner89-ATO}

\end{document}
